\documentclass[12pt, a4paper]{ltjsarticle}
\usepackage{luatexja, amsmath, amsthm, amssymb, graphicx}

\begin{document}

\thispagestyle{empty}
このページの番号は消した.

ブランクページが望まれた場合は空行でいい.
\newpage

\section*{はじめ}
\setcounter{page}{0}
このページは0にセットした.
これから数学・物理レポート用の\LaTeX テンプレートを作ってみたいと思う.
字下げは1文字にしてあるが, 
しばらく変更しないとする.

まずは書式. 
基本的に太字と下線以外, 英字のみ対応できる.

\textup{Basic}\ \texttt{Typewriter} \textsf{Sans Serif} \textit{Italic}\ \textsl{Slanted}\ \textbf{Bold}\ \textsc{Small Capitals} \underline{Underline}

\textup{基本}\ \textbf{太字}\ \underline{下線}
\[ \mathbb{MATHBBCPTL} \mathbf{MathBF} \mathsf{MathSF} \mathfrak{MathFrak} \mathcal{MATHCALCPTL} \mathsf{MathSF} \mathtt{MathTT} \]


\begin{itemize}
    \item 箇条(番号なし)
    \item 箇条
\end{itemize}

\begin{enumerate}
    \item 箇条(番号あり)
    \item 箇条
\end{enumerate}

普通の記号については,

\[``\ " (公式の外なら'') \dots \  \# \$ \% \& \_ \{ \}\]

のように書く.

% ~^については裏技を使わなければ表示できない

\setcounter{section}{-1}
\section{節・段落}
\texttt{$\backslash$setcounter}で -1 にセットしたので, 
この節は0になる. 
counterは, 現在の状態を変更する.

\subsection{Subsection}
Subsection 1

\subsection{Subsection}
Subsection 2

\section{数式}
\subsection{公式}
文章の中で公式を書くのは $E=mc^2$のように, 
また独立の行で

    \[ E=mc^2 \]

このように実現する.
またもう一つ独立な公式の書き方は

\setcounter{equation}{-1}
\begin{equation}
    1+1=2
\end{equation}

このようである. そこに番号はついてあり, 
同じくcounterでリセットできる.

\begin{equation}
    U_k=\frac{1}{2}mv^2
\end{equation}
\label{U_k}

(\ref{U_k})を引用する場合は, \texttt{$\backslash$label}を使う. 
この二つの様式を活用して, 見やすいレポートが作れる. 

\subsection{記号}
上付き$^{upper}$と下付き$_{lower}$はこのように書き, 
積分の上下限と総和記号にも使う.
また上付き文字と下付き文字も, それぞれ集合して

    \[ ^{upper\ left}_{j_1 j_2 \dots j_q}T^{i_1 i_2 \dots i_p}_{lower\ right} 
    = T(x^{i_1},\dots,x^{i_p},e_{j_1},\dots,e_{j_q}) \]

このようにも表現する.

積分などは$\int, \oint,\ \iint$と$\int_0^1 dx$で, 
分式は$\frac{a}{b}$である. 
偏微分の記号は$\partial$で, 
ベクトルは$\vec{A}$である.
ルートは$\sqrt{x}$と$\root n \of{x}$である.

総和記号と積記号は$\sum, \prod$と$\sum_{i=1}^n$で, 
極限は$\lim_{x\rightarrow\infty}$である.
これらの様式は, 異なる環境(\texttt{\textit{equation}}など)で違っている.

大型のかっこ(絶対値)は

    \[\left(\frac{\frac{a}{abc}}{abc}\right) \left[\frac{\frac{a}{abc}}{abc}\right] \left\{\frac{\frac{a}{abc}}{abc}\right\} \left|\frac{\frac{a}{abc}}{abc}\right|\]

である.

三角関数, 対数関数等特殊な関数は, 
\texttt{$\backslash$backslash}が必要となる: 
$\sin(\beta)$, $\cos(\alpha)$, $\log(x)$.
他の関数は, $\mbox{div} \vec{x}$のように書く.

ギリシャ小文字は$\alpha$, $\gamma$で, 
大文字は$\Delta$, $\Omega$などで書く.

運算記号については, 

    \[\pm \mp \times \div \cdot \backslash \setminus \ast \star \parallel \mid \propto \]
    \[\leq \geq \leqq \geqq \leqslant \geqslant \neq \simeq \cong \fallingdotseq \subset \supset \subseteq \supseteq \in \ni\]
    \[\leftarrow \longleftarrow \Leftarrow \Longleftarrow \rightarrow \longrightarrow \Rightarrow \Longrightarrow \leftrightarrow \longleftrightarrow \Leftrightarrow \Longleftrightarrow \uparrow \Uparrow \downarrow \Downarrow \updownarrow \Updownarrow \leftrightarrows \]
    \[\overrightarrow{overright} \overline{overline} \underline{underline} \overbrace{overbrace} \underbrace{underbrace} \binom{1}{2} \]

の通りである.

\subsection{行列, 連立方程式など}
\[
    \begin{pmatrix}
        x_{11} & \cdots & x_{1n} \\ 
        \vdots & \ddots & \vdots \\
        x_{m1} & \cdots & x_{mn} \\
    \end{pmatrix}
\]

\[
    \begin{vmatrix}
        x_{11} & \cdots & x_{1n} \\ 
        \vdots & \ddots & \vdots \\
        x_{m1} & \cdots & x_{mn} \\
    \end{vmatrix}
\]

\[
    |x|=
    \begin{cases}
        x & x \geq 0 \\
        -x & \text{文字でも説明できる.}
    \end{cases}  
\]

$\begin{cases}
    \ x^2&=y^2+z^2 \\
    \ x^3+y^3&<z^3+w^3
\end{cases}$\ \ 
$\begin{cases}
    \ x^2=y^2+z^2 \\
    \ x^3+y^3<z^3+w^3
\end{cases}$ 

\begin{align}
    x^2 &= y^2+z^2 \text{\ これも\texttt{\textit{equation}}の} \\
    x^3+y^3 &< z^3+w^3 \text{\ counterに従う.}
\end{align}

\newtheorem{teiri}{定理(系)}
\begin{teiri}
    定理などの項目序列は定義できる.
\end{teiri}

\end{document}