\documentclass[12pt, a4paper]{ltjsarticle}
\usepackage{luatexja, amsmath, amsthm, amssymb, graphicx}

\begin{document}
\thispagestyle{empty}

このページの番号は消した.

ブランクページが望まれた場合は空行でいい.

\newpage

\section*{はじめ}
\setcounter{page}{0}

このページは0にセットした.

これから数学・物理レポート用の\LaTeX テンプレートを作ってみたいと思う.
字下げは1文字にしてあるが, しばらく変更しないとする.

まずは書式. 基本的に太字と下線以外, 英字のみ対応できる.

\textup{Basic}\ \textit{Italics}\ \textsl{Forced italics}\ \textbf{Bold}\ \underline{Underline}

\textup{基本}\ \textbf{太字}\ \underline{下線}

\begin{itemize}
    \item 箇条(番号なし)
    \item 箇条
\end{itemize}

\begin{enumerate}
    \item 箇条(番号あり)
    \item 箇条
\end{enumerate}

句点については, ``コマ'', ドット\dots

\setcounter{section}{-1}
\section{節・段落}

setcounterで-1にセットしたのでこの節は0になる. 
counterは, 現在の状態を変更する.

\subsection{Subsection}

Subsection 1

\subsection{Subsection}

Subsection 2

\section{公式}

文章の中で公式を書くのは $E=mc^2$のように, 
また独立の行で

\[ E=mc^2 \]

このように実現する.

またもう一つ独立な公式の書き方は

\begin{equation}
1+1=2
\end{equation}

このようである. そこに番号はついてある.

\begin{equation}
    U_k=\frac{1}{2}mv^2
\end{equation}
\label{U_k}

引用する場合(\ref{U_k})は, ``label''を使う. 
この二つの様式を活用して, 見やすいレポートが作れる. 
また, 上付き文字と下付き文字も, それぞれ集合して

\[ ^{i_1 i_2 \dots i_p}_{j_1 j_2 \dots j_q}T^{i_1 i_2 \dots i_p}_{j_1 j_2 \dots j_q} 
= T(x^{i_1},\dots,x^{i_p},e_{j_1},\dots,e_{j_q}) \]

このようにも表現する.
 
積分は$\int$で, 分式は$\frac{a}{b}$. 上付き$^{upper}$と下付き$_{lower}$はこのように書き, 積分の上下限も同じ.

\[ \int_0^1 \frac{dx}{e^x} =  \frac{e-1}{e} \]

グリシア小文字は$\omega$, $\delta$で, 大文字は$\Omega$, $\Delta$で書く.

三角関数, 対数関数等特殊な関数は, backslashが必要となる: $\sin(\beta)$, $\cos(\alpha)$, $\log(x)$.

\end{document}