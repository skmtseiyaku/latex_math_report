\documentclass[12pt, a4paper]{ltjsarticle}
\usepackage{luatexja, amsmath, amsthm, amssymb, graphicx}

\begin{document}

\section*{0\ \ \ はじめ}

これから数学・物理レポート用の\LaTeX テンプレートを作ってみたいと思う.
字下げは1文字にしてあるが, しばらく変更しないとする.

まずは書式. 基本的に太字と下線以外, 英字のみ対応できる.

\textup{Basic}\ \textit{Italics}\ \textsl{Forced italics}\ \textbf{Bold}\ \underline{Underline}

\textup{基本}\ \textbf{太字}\ \underline{下線}

\begin{itemize}
    \item 箇条(番号なし)
    \item 箇条
\end{itemize}

\begin{enumerate}
    \item 箇条(番号あり)
    \item 箇条
\end{enumerate}

句点については, ``コマ'', ドット\dots 

\section{節・段落}

\subsection{Subsection}

Subsection 1

\subsection{Subsection}

Subsection 2

\section{公式}

文章の中で公式を書くのは $E=mc^2$のように, 
また独立の行で

\[ E=mc^2 \]

このように実現する.

またもう一つ独立な公式の書き方は

\begin{equation}
E=m
\end{equation}

このようである. 以下のように, そこに番号はついてある.

\begin{equation}
    U_k=\frac{1}{2}mv^2
\end{equation}
\label{U_k}

引用する場合(\ref{U_k})は, ``label''を使う. 
この二つの様式を活用して, 見やすいレポートが作れる. 
また, 上付き文字と下付き文字も, それぞれ集合して

\[ T^{i_1 i_2 \dots i_p}_{j_1 j_2 \dots j_q} = T(x^{i_1},\dots,x^{i_p},e_{j_1},\dots,e_{j_q}) \]

このようにも表現する.
 
We write integrals using $\int$ and fractions using $\frac{a}{b}$. Limits are placed on integrals using superscripts and subscripts:

\[ \int_0^1 \frac{dx}{e^x} =  \frac{e-1}{e} \]

Lower case Greek letters are written as $\omega$ $\delta$ etc. while upper case Greek letters are written as $\Omega$ $\Delta$.

Mathematical operators are prefixed with a backslash as $\sin(\beta)$, $\cos(\alpha)$, $\log(x)$ etc.

\end{document}